\chapter{Algorithms}\label{chapter:algorithms}
In the following chapter different approaches for generating small expansion sets $S$ as well as the edge expansion of hypergraphs are discussed.
%TODO: NP-hard? --> approximation needed
%todo: which tense to use where?



\section{Brute force}
One obvoius approach for generating the edge expansion $\Phi(H)$ of a hypergraph $H$ is to brute-force the problem like in \cref{alg:brute_force}.
\begin{algorithm}[H]
	\caption{Brute-force edge expansion on a hypergraph \label{alg:brute_force}}
	\begin{algorithmic}
		\Function{BruteForceEdgeExpansion}{$H := (V,E, w)$}
		\State $best\_S := null$
		\State $lowest\_expansion := \infty$
		\For{$\emptyset \neq S \subsetneq V$}
		\State $expansion :=  \max{ \Phi(S), \Phi({V\setminus S})}$
		\If{$expansion < lowest\_expansion$}
			\State $lowest\_expansion := expansion$
			\State $best\_S := S$
		\EndIf
		\EndFor	
		\State return $best\_S$
		\EndFunction
	\end{algorithmic}
\end{algorithm}


As \cref{alg:brute_force} iterates over all $\emptyset \neq S \subsetneq V$, it computes $\arg\min_{\emptyset \subsetneq S \subsetneq V} \max{( \Phi(S), \Phi({V\setminus S}))}$.

There are $2^{|V|}-2 = 2^{n}-2 \in O(2^n) $ combinations for $\emptyset \neq S \subsetneq V$, namely all the $2^{|V|}$ subsets of $V$ excluding the empty set $\emptyset$ and $V$ itself. Hence, this algorithm is of exponential time complexity in $n$ and is therefore not efficient for larger graphs as evaluated in ....
%TODO: In fact the problem is  unique games conjecture, np-hard

For the purpose of analyzing the graph creation algorithms in \cref{chapter:random_hypergraphs}, it can be insightful to observe the lowest expansion of each possible size (in the number of vertices) like in \cref{alg:brute_force_size}.

\begin{algorithm}[H]
	\caption{Brute-force edge expansion of a hypergraph for every size   \label{alg:brute_force_size}}
	\begin{algorithmic}
		\Function{BruteForceEdgeExpansionSizes}{$H := (V,E, w)$}
		\State $best\_S_of_size := \{\}$
		\State $lowest\_expansion\_of\_size := \{1:\infty, 2:\infty, \ldots, n-1: \infty \}$
		\For{$\emptyset \neq S \subsetneq V$}
		\State $expansion := \max{ \Phi(S), \Phi({V\setminus S})}$
		\If{$expansion < lowest\_expansion\_of\_size[|S|]$}
		\State $ lowest\_expansion\_of\_size[|S|] := expansion$
		\State $best\_S\_of\_size[|S|] := S$
		\EndIf
		\EndFor	
		\State return $best\_S\_of\_size$
		\EndFunction
	\end{algorithmic}
\end{algorithm}


In order to analyze the results from \cref{alg:ses}, which only computes $\Phi(S)$, the expansion of a set $S$, not the whole graph, it makes sense to compare it with the best result possible for the same size of $S$. Therefore, just the line for the computation of the expansion \cref{alg:brute_force_size} needs to be changed to get \cref{alg:brute_force_size_just_one_side}.

\begin{algorithm}[H]
	\caption{Brute-force edge expansion of sets for every size   \label{alg:brute_force_size_just_one_side}}
	\begin{algorithmic}
		\Function{BruteForceEdgeExpansionSizes}{$H := (V,E, w)$}
		\State $best\_S_of_size := \{\}$
		\State $lowest\_expansion\_of\_size := \{1:\infty, 2:\infty, \ldots, n-1: \infty \}$
		\For{$\emptyset \neq S \subsetneq V$}
		\State $expansion :=  \Phi(S)$
		\If{$expansion < lowest\_expansion\_of\_size[|S|]$}
		\State $ lowest\_expansion\_of\_size[|S|] := expansion$
		\State $best\_S\_of\_size[|S|] := S$
		\EndIf
		\EndFor	
		\State return $best\_S\_of\_size$
		\EndFunction
	\end{algorithmic}
\end{algorithm}
For \cref{alg:brute_force_size} and \cref{alg:brute_force_size_just_one_side} the above argument of for exponential complexity holds as well.

\section{Orthonormal vectors}
As described in \cite{ChanLTZ16}, an algortihm for generating a random small expansion set can be derived. For that, a set of orthonormal vectors $\{f_1, \ldots, f_k\}$ needs to be created, where each vector $f_i \in \mathbb{R}^V$ gives a value to each vertex. Because of \cref{fact:small_xi} it will be of importance for \cref{alg:small_set_expansion} that  $\xi := \max_{s \in [k]} D_w(f_s)   $ is small.





\begin{algorithm}[H]
	\caption{Small Set Expansion (according to Algorithm 1 in \cite{ChanLTZ16}) \label{alg:small_set_expansion}} %todo: better name
	
	
	\begin{algorithmic}
		\Function{SmallSetExpansion}{$G := (V,E, w), f_1, \ldots , f_k$}
		\State assert $\xi == \max_{s\in [k]} \{D_w(f_s)\}$
		\State assert $\forall f_i, f_j \in \{f_1, \ldots , f_k\} \subset \mathbb{R}^n, i\neq j: f_i \text{ and } f_j \text{ orthonormal in weighted space} $
		
		\For{$i \in V$}
		\For{$s\in [k]$}
		\State	$u_i(s) := f_s(i) $
		\EndFor
		\EndFor
		
		\For{$i \in V$}
		\State $\tilde{u}_i := \frac{u_i}{||u_i||}$
		\EndFor
		
		\State $\hat{S} := $ \Call{OrthogonalSeparator}{$\{\tilde{u}_i\}_{i\in V} , \beta = \frac{99}{100}, \tau = k$ }
		
		\For {$i \in S$}
		\If {$\tilde{u}_i \in \hat{S}$ }

		\State $X_i := ||u_i||^2$
		\Else
		\State $X_i := 0$
		\EndIf
		
		\EndFor
		\State $X:= $ sort $ list(\{X_i\}_{i \in V})$
		\State $V := [i]_{\text{in order of X}}$
		\State $S := \arg \min_{\{P: O \text{ is prefix of }V\}}\phi(O)$
		
		\State return $S$
		
	
	
		\EndFunction
		
		
	\end{algorithmic}
\end{algorithm} %todo: explain list syntax in notation


\begin{algorithm}[htpb]
	\caption{Orthogonal Separator (combination of Lemma 18 and algorithm Theorem 10 in \cite{LouisM14} (also Fact 6.7 in \cite{ChanLTZ16})) \label{alg:orthogonal_separator}} 
\begin{algorithmic}
	\Function{OrthogonalSeparator}{$\{\tilde{u}_i\}_{i\in V} , \beta = \frac{99}{100}, \tau = k$}
	\State $l := \lceil \frac{\log_2 k}{1-\log_2 k}\rceil$
	\State $g \sim \mathcal{N} (0,I_n)$ where each component $g_i$ is mutually independent and sampled from $\mathcal{N} (0,1)$
	

	
	\State $w := $\Call{SampleAssignments}{$l, V, \beta$}
	
	\For{$ i \in V$}
	\State $W(u) := w_1(u)w_2(u)\cdots w_j(u)$
	\EndFor
	
	\If{$n\ge 2^l$}
	\State $word := random( \{0,1\}^l)$ uniform
	
	\Else
	 
	\State $words := set({w(i): i\in V})$  no multiset
	\State $words \cup= \{w_1, \ldots , w_{|V|-|words|} \in \{0,1\}^l\} $ random choice
	\State $word := random(words)$ uniform
	
	\EndIf
	
	\State $r := uniform(0,1)$
	\State $S := \{i \in V: ||i||^2 \ge r \land W(u) = word \}$
	\State return $S$
	
	\EndFunction %todo: 1-vector explain;
	%todo: explain norm
	%todo: i or u or v for what?
\end{algorithmic}
\end{algorithm}	

	
	
\begin{algorithm}[htpb]
	\caption{Sample Assignments (proof of Lemma 18 in \cite{LouisM14}) \label{alg:sample_assignments}} 
	\begin{algorithmic}
	\Function{SampleAssignments}{$l, V, \beta$}
		\State $\lambda := \frac{1}{\sqrt{\beta}}$ 
		\For {$j = 1, 2, \ldots, l$}
	\For{$i\in V$}
	
	
	\State $t_i := \langle g, \tilde{u}_i \rangle $
	\State $poisson\_count_i := N(t_i, \lambda)$ where N is a poisson process on $\mathbb{R}$ (same process for each call)
	\If{$poisson\_count_i \mod 2 == 0 $}
	\State $w_j(i) := 1$
	\Else
	\State  $w_j(i) := 0$
	\EndIf
	\EndFor
	\EndFor
	\State return $w$
	\EndFunction %todo: 1-vector explain
\end{algorithmic}
\end{algorithm}	
\begin{fact}{(Theorem 6.6 in \cite{ChanLTZ16})}\label{fact:small_xi}
	Given a hypergraph $H = (V, E, w)$ and $k$ vectors $f_1, f_2, \ldots , f_k$ which are orthonormal in the weighted space with $ \max_{s \in [k]} D_w(f_s) \le \xi $, the following holds: Algorithm \ref{alg:small_set_expansion} constructs a random set $S \subsetneq V$ in polynomial time such that with $\Omega(1)$ probability, $|S| \le \frac{24|V|}{k}$ and
	 \begin{equation}\label{eq:small_expansion}
	 \phi(S) \le C \min\{\sqrt{r \log k}, k \log k  \log \log k \sqrt{\log r} \} \cdot \sqrt{\xi},
	 \end{equation}
	 where $C$ is an absolute constant and $r := \max_{e\in E} |e|$.
\end{fact}

With \cref{alg:sdp} to create orthonormal vectors, algorithm \cref{alg:ses} can be executed. This combination results in \cref{alg:ses} according to ...



\begin{mini}
	{g}{\text{SDPval} := \sum_{e\in E} w_e \max_{u,v\in e} ||\vec{g_u} -\vec{g_v} ||^2}{}{}
	\addConstraint{ \sum_{u\in V} w_v ||\vec{g_v}||^2 }{= 1}{}
	\addConstraint{ \sum_{u\in V} w_v f_i(v)  \vec{g_v} }{=\vec{0},\quad}{\forall i \in [k-1]}
	\label{SDP}
\end{mini}


SDP

\begin{algorithm}[htpb]
	\caption{Rounding Algorithm for Computing Eigenvalues (Algorithm 3 in \cite{ChanLTZ16}) \label{alg:sdp}} 
	\begin{algorithmic}
		\Function{SampleRandomVector}{$H, f_1, \ldots, f_{k-1}$}
		\State Solve SDP \ref{SDP} to generate vectors $\vec{g_v} \in \mathbb{R}^n $ for $v \in V$
		\State $\vec{z} := sample(\mathcal{N}(0,1^n))$
		\For{$v\in V $}
		\State $f(v) := <\vec{g_v}, \vec{z}>$
		\EndFor
		\State return $f$
		\EndFunction 
	\end{algorithmic}
\end{algorithm}	

\begin{algorithm}[htpb]
	\caption{Procedural Minimizer \label{alg:sdp}} 
	\begin{algorithmic}
		\Function{SampleSmallVectors}{$H,k$}
		\State $f_:1= \frac{\vec{1}}{||\vec{1}||_w}$
	
		\For{$i = 2, \ldots, k$}
		\State $f_i := $\Call{SampleRandomVector}{$H, f_1, \ldots, f_{i-1}$}
		\EndFor
		\State return $f_1, \ldots , f_k$
		\EndFunction 
	\end{algorithmic}
\end{algorithm}	

So for a given Graph $H$ we can find a small expansion set with the following algorithm: 
\begin{algorithm}[htpb]
	\caption{Find Small Expansion Set \label{alg:ses}} 
	\begin{algorithmic}
		\Function{SES}{$H$}
		\State $f := $\Call{SampleRandomVectors}{$H$}
		\State return \Call{SmallSetExpansion}{$H, f$}
		\EndFunction 
	\end{algorithmic}
\end{algorithm}	

%todo: Algorithm instead of algorithm at sentence-beginning
%todo: formatting


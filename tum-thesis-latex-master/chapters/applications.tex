\chapter{Applications}\label{chapter:applications}
Hypergraphs can be used to model social networks in which users interact with each other. One could represent the users as vertices and the hyperedges as interactions between users. If for example ten users discuss a post with each other, they would receive an edge with a weight depending on the intensity of the interaction. If for example, interest lies in finding a group of close friends, which interact mostly and most intensely with each other, one can use the approximation algorithm, \cref{alg:ses} in order to find such a group. The algorithm would be called with a k adjusted to the total number of users and the favoured number of users. Further discussion of applications like these can be found in\cite{zhang2010hypergraph}.

For further applications of random hypergraphs, the reader is referred to \cite{ghoshal2009random}.

One other possible application of the discussed algorithm might be to solve puzzle games like Rummikub\footnote{\href{https://en.wikipedia.org/wiki/Rummikub}{https://en.wikipedia.org/wiki/Rummikub}}. There the stones would be the vertices in a hypergraph and the edges and their weights would indicate how well specific stones can be combined with another. As only some of the stones are visible usually, in order to win the game it is required to find a new combination which can take in some of the private stones of each player. A small expansion set could be a good heuristics for starting the search for new combinations. Admittedly, the possible actions in the game are limited but results might be also used in other matching-like problems.


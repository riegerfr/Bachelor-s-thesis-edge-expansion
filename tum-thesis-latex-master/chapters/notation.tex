\chapter{Notation}\label{chapter:notation}

The notation used in this thesis is orientated on \cite{ChanLTZ16}.



The weight matrix can be denoted as \begin{equation}
	W = 
	\begin{pmatrix}
	w_{v_1} & 0 & 0&\dots &0 \\
	0 & w_{v_2} & 0 & \ldots & 0 \\
	0 & 0 & w_{v_3} & \ldots & 0 \\
	\vdots & \vdots & \vdots & \ddots & \vdots \\
	0 &0&0& \ldots  & w_{v_n}
	\end{pmatrix} \in \mathbb{R}_{0+}^{n \times n} .
\end{equation} 
% TODO: needed?

The discrepancy ratio of a graph, given a non-zero vector $f \in \mathbb{R}^V$ is defined as \begin{equation}\label{eq:discrepancy_ratio}
D_w(f) := \frac{\sum_{e\in E} w_e \max_{u,v\in e}(f_u - f_v)^2}{\sum_{u\in V} w_u f_u^2}.
\end{equation} Observe that $0\le D_w(f) \le 2 $ \cite{ChanLTZ16}.

In the weighted space, in which the discrepancy ratio is defined like above, for two vectors $f, g \in \mathbb{R}^V$ the inner product is defined as $ \langle f,g \rangle_w := f^T W g$. Accordingly, the norm is $||f||_w = \sqrt{ \langle f,f \rangle_w}$.
If $ \langle f,g \rangle_w   = 0 $, $f$ and $g$ are said to be orthonormal in the weighted space.


\begin{equation}
\mathcal{D}(x) = D_w(W^{-\frac{1}{2}}x)
\end{equation}
\begin{equation}\label{eq:xi}
	\xi_k := \min_{x_1, \ldots , x_k} \max_{i \in [k]} \mathcal{D}(x_i)
\end{equation}


Todo explain minimaximizer, Discrepancy ration in ... space, gamma 2
Todo: weighted in which space orthogonal
Laplacian? Eigenvalues?

%TODO: small/big correct words?

%TODO: New words in italics?

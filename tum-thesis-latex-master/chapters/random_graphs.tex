\chapter{Random Hypergraphs}\label{chapter:random_hypergraphs}


The initial intention for creating random r-uniform, d-regular, connected hypergraphs in an effective manner which is guaranteed to terminate showed to be not as a trivial task, which is why several different approaches will be used and their resulting graphs also characterized by their edge expansion

To start with, a simple algorithm to generate graphs which follows some of the intentions will be discussed:
\begin{algorithm}[h!]
	\caption{Generate random graph} 
	\begin{algorithmic}
		\Function{GenerateRandomGraph}{$n, rank, numberEdges, weightDistribution$}
		\State $E := \emptyset$
		\State $V := \{v_1, \ldots, v_n\}$
		\For{$1, \ldots , numberEdges$}
		\State $nextEdgeVertices := sample(V, rank) $ \Comment draw without replacement
		\State $nextEdgeWeight := sample(weightDistribution)$ 
		\State $nextEdge := (nextEdgeVertices, nextEdgeWeight)$
		\State $E := E \cup \{nextEdge\}$
		\EndFor
		\EndFunction 
	\end{algorithmic}
\end{algorithm}	

The algorithm is terminating, quick and the resulting graph is r-uniform and all the possible graphs can be constructed.
But it might never sample one vertex $v \in V$, therefore the rank of this vertex will be 0 which does not make the graph regular and also not connected. Also, it does not guarantee to have no doubled edges.

This idea can be improved by ensuring the degree of the vertices do not exceed $d$:

\begin{algorithm}[h!]
	\caption{Generate random graph} 
	\begin{algorithmic}
		\Function{GenerateRandomGraph}{$n, rank, d, weightDistribution$}
		\State $E := \emptyset$
		\State $V := \{v_1, \ldots, v_n\}$
		\While{$|\{v\in V| deg(v)< d\}| \ge rank$}
		\State $nextEdgeVertices := sample(\{v\in V| deg(v)< d\}, rank) $ \Comment draw without replacement
		\State $nextEdgeWeight := sample(weightDistribution)$ 
		\State $nextEdge := (nextEdgeVertices, nextEdgeWeight)$
		\State $E := E \cup \{nextEdge\}$
		\EndWhile
		\EndFunction 
	\end{algorithmic}
\end{algorithm}	

The algorithm is terminating, quick and the resulting graph is r-uniform and all the possible graphs can be constructed.
However it is not guaranteed that this graph is connected and it is possible that some ($< rank$) vertices do not have degree $d$ in the end, because they have not been sampled before. TODO: example
Also, it does not guarantee to have no doubled edges.

To overcome these problems, the edges could only be sampled from the vertices with the smallest degrees:

\begin{algorithm}[h!]
	\caption{Generate random graph} 
	\begin{algorithmic}
		\Function{GenerateRandomGraph}{$n, rank, d, weightDistribution$}
		\State $E := \emptyset$
		\State $V := \{v_1, \ldots, v_n\}$
		\While{$|\{v\in V| deg(v)< d\}| \ge rank$}
		\State $ smallestDegreeVertices := \{v\in V| deg(v) = \min_{u\in V} deg(u) \}$
		\If {$|smallestDegreeVertices| >= rank$}
		\State $nextEdgeVertices := sample(smallestDegreeVertices, rank) $ \Comment draw without replacement
		\Else
		\State $secondSmallestDegreeVertices := \{v\in V| deg(v) = \min_{u\in V} deg(u) +1 \}$
		\State $nextEdgeVertices :=  sample(secondSmallestDegreeVertices, rank - | smallestDegreeVertices|)$
		\State $nextEdgeVertices := smallestDegreeVertices \cup nextEdgeVertices  $
		\EndIf
		\State $nextEdgeWeight := sample(weightDistribution)$ 
		\State $nextEdge := nextEdgeVertices$
		\State $E := E \cup \{nextEdge\}$
		\State $w(e):= nextEdgeWeight$
		
		\EndWhile
		\State return $G:=(V,E, w)$	
		\EndFunction 
	\end{algorithmic}
\end{algorithm}	

The algorithm is guaranteed to terminate, quick and the resulting graph is r-uniform and d-regular.
However, not all the possible graphs can be constructed: This algorithm basically constructs the edges by d r-matchings. But not every graph can be dissembled into d r-matchings.
Again this graph is not neccessarily connected and some edges might be doubled.

To solve this, there are several options:
One could i) resample whole graph (if probability is > constant), losing the terminating property.
ii) resample some edges (from different connection components), ideally only strongly connected vertices, also losing the terminating property. Proof: all vertices are strongly connected to their connection component?
iii) creating a spanning tree first and then sampling further

i)
\begin{algorithm}[h!]
	\caption{Generate random graph} 
	\begin{algorithmic}
		\Function{GenerateRandomGraph}{$n, rank, d, weightDistribution$}
		\State $G:=$ \Call{GenerateRandomGraph}{$n, rank, d, weightDistribution$}
		\While{$\not$ Connected(G) or $ \exists e,f \in E. e = f$}
		\State $G:=$ \Call{GenerateRandomGraph}{$n, rank, d, weightDistribution$}
		\EndWhile
		\State return $G:=(V,E)$	
		\EndFunction 
	\end{algorithmic}
\end{algorithm}	

ii)

\begin{algorithm}[h!]
	\caption{Generate random graph} 
	\begin{algorithmic}
		\Function{GenerateRandomGraph}{$n, rank, d, weightDistribution$}
		\State $G:=$ \Call{GenerateRandomGraph}{$n, rank, d, weightDistribution$}
		\While{$\not$ Connected(G) or $ \exists e,f \in E. e = f$}
		\State $e,f := sample(E, 2)$
		\State $u := sample(e)$
		\State $v := sample(f)$
		\State $e := (e \cup \{v\}) \setminus \{u\}$
		\State $f := (f \cup \{u\}) \setminus \{v\}$
		\EndWhile
		\State return $G:=(V,E, w)$	
		\EndFunction 
	\end{algorithmic}
\end{algorithm}	


iii)

\begin{algorithm}[h!]
	\caption{Generate random graph} 
	\begin{algorithmic}
		\Function{GenerateRandomGraph}{$n, rank, d, weightDistribution$}
		\State $V := \{v_1, \ldots, v_n\}$
		\State $E := choice(V,rank)$
		\While {$\{v\in V| deg(v) = 0 \} \neq \emptyset$}
		\If{ $|\{v\in V| deg(v) = 0 \}| \ge rank$}
			\State $nextEdgeTreeVertex := choice(\{v\in V| deg(v) = 1 \})$\Comment get one tree node
			\State $nextEdgeVertices := choice(\{v\in V| deg(v) =0\}, rank-1) \cup \{nextEdgeTreeVertex\}$
			\Else
			\State  $nextEdgeVertices :=  \{v\in V| deg(v) =0\},  \cup choice(\{v\in V| deg(v) >0\},| \{v\in V| deg(v) =0\}| ) $
			\EndIf
			\State $nextEdgeWeight := sample(weightDistribution)$ 
			\State $nextEdge := nextEdgeVertices$
			\State $E := E \cup \{nextEdge\}$
			\State $w(e):= nextEdgeWeight$
			\EndWhile
			\While{$|\{v\in V| deg(v)< d\}| \ge rank$}
			\State $ smallestDegreeVertices := \{v\in V| deg(v) = \min_{u\in V} deg(u) \}$
			\If {$|smallestDegreeVertices| >= rank$}
			\State $nextEdgeVertices := sample(smallestDegreeVertices, rank) $ \Comment draw without replacement
			\Else
			\State $secondSmallestDegreeVertices := \{v\in V| deg(v) = \min_{u\in V} deg(u) +1 \}$
			\State $nextEdgeVertices :=  sample(secondSmallestDegreeVertices, rank - | smallestDegreeVertices|)$
			\State $nextEdgeVertices := smallestDegreeVertices \cup nextEdgeVertices  $
			\EndIf
			\State $nextEdgeWeight := sample(weightDistribution)$ 
			\State $nextEdge := nextEdgeVertices$
			\State $E := E \cup \{nextEdge\}$
			\State $w(e):= nextEdgeWeight$
			\EndWhile
			\State return $G:=(V,E, w)$	
			\EndFunction 
	\end{algorithmic}
\end{algorithm}	
	
However it is not guaranteed that this graph is connected and it is possible that some ($< rank$) vertices do not have degree $d$ in the end, because they have not been sampled before.


TODO: define quick

TODO: Discuss different approaches of generating, their limitations

TODO: Analyze $\Phi$ for different random- classes? (and explain?)

\section{random edges with discard if not connected}




\section{connected edges with discard if not connected or not regular}
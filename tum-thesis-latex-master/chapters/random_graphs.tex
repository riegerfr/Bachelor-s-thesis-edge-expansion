\chapter{Random Hypergraphs}\label{chapter:random_hypergraphs}
In order to evaluate the algorithms of \cref{chapter:algorithms} hypergraphs are require as inputs. However, instead of creating a few hypergraphs by hand, they shall be randomly generated in order to have a diverse array of graphs.

The initial intention for creating random r-uniform, d-regular, connected hypergraphs with no doubled edges in an effective manner which is guaranteed to terminate showed to be a non-trivial trivial challenge. Thetrefore, several different approaches which fulfill some of these cruteria will be discussed and their resulting graphs shall also be analyzed by their edge expansion.


\begin{algorithm}[htpb]
	\caption{Generate simple random graph\label{alg:simple_random_graph}} 
	\begin{algorithmic}
		\Function{GenerateRandomGraph}{$n, rank, numberEdges, weightDistribution$}
		\State $E := \emptyset$
		\State $V := \{v_1, \ldots, v_n\}$
		\State $w = \{\}$
		\For{$1, \ldots , numberEdges$}
		\State $nextEdge := sample(V, rank) $ \Comment draw without replacement
		\State $E := E \uplus	 \{nextEdge\}$
		\State $weight[nextEdge] := sample(weightDistribution)$ 
		\EndFor
		\State return $H = (V, E, w)$
		\EndFunction 
	\end{algorithmic}
\end{algorithm}	
To start with, a simple algorithm to generate graphs which follows some of the intentions shall be discussed. Algorithm \ref{alg:simple_random_graph} is terminating, of polynomial time complexity and the resulting graph is r-uniform and all the possible graphs can be constructed.
But it might never sample one vertex $v \in V$, therefore the rank of this vertex would be $0$ which does make the graph possibly non-regular and also not connected. Also, the algorithm does not guarantee to have no doubled edges.



\begin{algorithm}
	\caption{Generate random graph with upper bound on degrees\label{alg:GenerateRandomGraphBoundDegrees}} 
	\begin{algorithmic}
		\Function{GenerateRandomGraphBoundDegrees}{$n, rank, d, weightDistribution$}
		\State $E := \emptyset$
		\State $V := \{v_1, \ldots, v_n\}$
		\State $w = \{\}$
		\While{$|\{v\in V| deg(v)< d\}| \ge rank$}
		\State $nextEdge := sample(\{v\in V| deg(v)< d\}, rank) $ \Comment draw without replacement
		\State $E := E \uplus \{nextEdge\}$
		\State $weight[nextEdge] := sample(weightDistribution)$ 
		\EndWhile
		\State  return $H = (V, E, w)$
		\EndFunction
	\end{algorithmic}
\end{algorithm}	

This idea can be improved by ensuring the degree of the vertices do not exceed $d$ like in \cref{alg:GenerateRandomGraphBoundDegrees}. This algorithm is terminating, quick and the resulting graph is r-uniform and all the possible graphs can be constructed.
However it is not guaranteed that this graph is connected and it is possible that some ($< rank$) vertices do not have degree $d$ in the end, because they have not been sampled before. An example of such a situation can be seen in \cref{fig:exapmle_non_connected_uniform_hypergraph}.
Also, it does not guarantee to have no doubled edges.


\begin{figure} 
	\centering
	\begin{tikzpicture}
	\node[vertex,label=below:\(v_1\)] (v1) {};
	\node[vertex,right of=v1,label=below:\(v_2\)] (v2) {};
	\node[vertex,below of=v2,label=below:\(v_3\)] (v3) {};
	\node[vertex,left of=v3,label=below:\(v_4\)] (v4) {};	

	\begin{pgfonlayer}{background}
	\draw[edge,color=yellow] (v1) -- (v2);
	\draw[edge,color=red] (v2) -- (v3);
	\draw[edge,color=green] (v3) -- (v1) ;
	

	
	
	\end{pgfonlayer}
	
	\node[elabel,color=yellow,label=right: {$e_1 , w_{e_1} = 0.7$}]  (e1) at (-4,0) {};
	
	\node[elabel,below of=e1,color=red,label=right:{$e_2, w_{e_2} = 1.3$}]  (e2) {};
	\node[elabel,below of=e2,color=green,label=right:{$e_3, w_{e_3} = 1.5$}]  (e3) {};
	
	\end{tikzpicture}
	\caption[Example non connected uniform hypergraph]{An example for a non connected 2-uniform hypergraph which could have been created by \cref{alg:GenerateRandomGraphBoundDegrees}. }\label{fig:exapmle_non_connected_uniform_hypergraph}
\end{figure}



\begin{algorithm}
	\caption{Generate random hypergraph, sampling from lowest degrees\label{alg:randomHypergraphSmallestDegrees}} 
	\begin{algorithmic}
		\Function{GenerateRandomGraph}{$n, rank, d, weightDistribution$}
		\State $E := \emptyset$
		\State $V := \{v_1, \ldots, v_n\}$
		\While{$|\{v\in V| deg(v)< d\}| \ge rank$}
		\State $ smallestDegreeVertices := \{v\in V| deg(v) = \min_{u\in V} deg(u) \}$
		\If {$|smallestDegreeVertices| >= rank$}
		\State $nextEdgeVertices := sample(smallestDegreeVertices, rank) $ \Comment draw without replacement
		\Else
		\State $secondSmallestDegreeVertices := \{v\in V| deg(v) = \min_{u\in V} deg(u) +1 \}$
		\State $nextEdgeVertices :=  sample(secondSmallestDegreeVertices, rank - | smallestDegreeVertices|)$
		\State $nextEdgeVertices := smallestDegreeVertices \cup nextEdgeVertices  $
		\EndIf
		\State $nextEdgeWeight := sample(weightDistribution)$ 
		\State $nextEdge := nextEdgeVertices$
		\State $E := E \cup \{nextEdge\}$
		\State $w(e):= nextEdgeWeight$
		
		\EndWhile
		\State return $G:=(V,E, w)$	
		\EndFunction 
	\end{algorithmic}
\end{algorithm}	
To overcome these problems, the edges could only be sampled from the vertices with the smallest degrees like in \cref{alg:randomHypergraphSmallestDegrees}
The algorithm is guaranteed to terminate, quick and the resulting graph is r-uniform and d-regular.
However, not all the possible graphs can be constructed: This algorithm basically constructs the edges by d r-matchings. But not every graph can be dissembled into d r-matchings.
Again this graph is not neccessarily connected and some edges might be doubled.

To solve this, there are again several options:
Algorithm \ref{alg:GenerateRandomGraphWithResampling}  resamples the whole graph, if the .... and ... conditions are not met. This way, the algorithm loses the property of guaranteed terminating. The time complexity would depend on the probability of creating a graph which fulfills the requirements. However, this shall not be analyzed here.


This only works, if the probability for meeting the conditions are bigger than some constant, regardless of the parameters.
(if probability is > constant), losing the terminating property.
ii) resample some edges (from different connection components), ideally only strongly connected vertices, also losing the terminating property. Proof: all vertices are strongly connected to their connection component?
iii) creating a spanning tree first and then sampling further

i)
\begin{algorithm}[htpb]
	\caption{Generate random graph with resampling\label{alg:GenerateRandomGraphWithResampling}} 
	\begin{algorithmic}
		\Function{GenerateRandomGraph}{$n, rank, d, weightDistribution$}
		\State $G:=$ \Call{GenerateRandomGraph}{$n, rank, d, weightDistribution$}
		\While{$\not$ Connected(G) or $ \exists e,f \in E. e = f$}
		\State $G:=$ \Call{GenerateRandomGraph}{$n, rank, d, weightDistribution$}
		\EndWhile
		\State return $G:=(V,E)$	
		\EndFunction 
	\end{algorithmic}
\end{algorithm}	

ii)

\begin{algorithm}[htpb]
	\caption{Generate random graph} 
	\begin{algorithmic}
		\Function{GenerateRandomGraph}{$n, rank, d, weightDistribution$}
		\State $G:=$ \Call{GenerateRandomGraph}{$n, rank, d, weightDistribution$}
		\While{$\not$ Connected(G) or $ \exists e,f \in E. e = f$}
		\State $e,f := sample(E, 2)$
		\State $u := sample(e)$
		\State $v := sample(f)$
		\State $e := (e \cup \{v\}) \setminus \{u\}$
		\State $f := (f \cup \{u\}) \setminus \{v\}$
		\EndWhile
		\State return $G:=(V,E, w)$	
		\EndFunction 
	\end{algorithmic}
\end{algorithm}	


iii)

\begin{algorithm}[htpb]
	\caption{Generate random graph} 
	\begin{algorithmic}
		\Function{GenerateRandomGraph}{$n, rank, d, weightDistribution$}
		\State $V := \{v_1, \ldots, v_n\}$
		\State $E := choice(V,rank)$
		\While {$\{v\in V| deg(v) = 0 \} \neq \emptyset$}
		\If{ $|\{v\in V| deg(v) = 0 \}| \ge rank$}
			\State $nextEdgeTreeVertex := choice(\{v\in V| deg(v) = 1 \})$\Comment get one tree node
			\State $nextEdgeVertices := choice(\{v\in V| deg(v) =0\}, rank-1) \cup \{nextEdgeTreeVertex\}$
			\Else
			\State  $nextEdgeVertices :=  \{v\in V| deg(v) =0\},  \cup choice(\{v\in V| deg(v) >0\},| \{v\in V| deg(v) =0\}| ) $
			\EndIf
			\State $nextEdgeWeight := sample(weightDistribution)$ 
			\State $nextEdge := nextEdgeVertices$
			\State $E := E \cup \{nextEdge\}$
			\State $w(e):= nextEdgeWeight$
			\EndWhile
			\While{$|\{v\in V| deg(v)< d\}| \ge rank$}
			\State $ smallestDegreeVertices := \{v\in V| deg(v) = \min_{u\in V} deg(u) \}$
			\If {$|smallestDegreeVertices| >= rank$}
			\State $nextEdgeVertices := sample(smallestDegreeVertices, rank) $ \Comment draw without replacement
			\Else
			\State $secondSmallestDegreeVertices := \{v\in V| deg(v) = \min_{u\in V} deg(u) +1 \}$
			\State $nextEdgeVertices :=  sample(secondSmallestDegreeVertices, rank - | smallestDegreeVertices|)$
			\State $nextEdgeVertices := smallestDegreeVertices \cup nextEdgeVertices  $
			\EndIf
			\State $nextEdgeWeight := sample(weightDistribution)$ 
			\State $nextEdge := nextEdgeVertices$
			\State $E := E \cup \{nextEdge\}$
			\State $w(e):= nextEdgeWeight$
			\EndWhile
			\State return $G:=(V,E, w)$	
			\EndFunction 
	\end{algorithmic}
\end{algorithm}	
	
However it is not guaranteed that this graph is connected and it is possible that some ($< rank$) vertices do not have degree $d$ in the end, because they have not been sampled before.


\begin{table}[htpb]
  \caption[Graph creation algorithms comparison]{Comparison the properties of the graphs by the creation algorithms.}\label{tab:GraphCreationAlgorithmsComparison}
  \centering
  \begin{tabular}{l| l|l}
    
      property \ Algorithm & \ref{alg:GenerateRandomGraphBoundDegrees}& 1  \\
    \midrule
      d-regular &no & yes\\
      r-uniform & yes\\
     no doubled edges & \\
     	connected &\\
     	terminating &\\
     	polynomial time complexity&\\
     	 all possible graphs &&\\
  \end{tabular}
\end{table}

random graph model:\cite{ghoshal2009random, zhang2010hypergraph}
TODO: define quick

TODO: Discuss different approaches of generating, their limitations

TODO: Analyze $\Phi$ for different random- classes? (and explain?)

\section{random edges with discard if not connected}




\section{connected edges with discard if not connected or not regular}
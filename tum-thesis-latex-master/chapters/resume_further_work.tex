\chapter{Conclusion and Further Work}\label{chapter:resmue_further_work}

All in all, approximation algorithms for creating small expansion sets showed to be a stimulating topic from a theoretical perspective. However, they also have interesting applications in the real world as seen in \cref{chapter:applications}. In this thesis, it is shown in \cref{chapter:algorithms} that by making use of the spectral properties of hypergraphs, through creating orthogonal vectors, using an SDP, one can find sets of strongly connected vertices with a low edge expansion. This was verified with an implementation which is described in \cref{chapter:implementation} and the results are evaluated in \cref{chapter:Evaluation}.
Additionally, finding an algorithm for creating random hypergraphs which fulfill certain properties \cref{chapter:random_hypergraphs} proved to be appetizing for further research.

The biggest challenge in this thesis were the extraction and understanding of the approximation algorithm and the algorithms it depends on. Furthermore, developing algorithms for the creation of random graphs with the placed properties demonstrated to be tricky. During implementation, especially finding a suitable optimizer for the SDP and tuning the right parameters for it not to get stuck were challenging.

Several aspects for further work can be noted. 
For one, the efficiency of the implementation can be improved. As the bottleneck of the implementation is the SDP optimization, biggest improvements can be achieved there. One could try adjusting the options for the current optimizer, try different optimizers altogether and as well speed up the calculation of the $SDPvalue$ and the constraints. As of now, for better overview, these calculations require accessing different Python dictionaries, which could be handled more efficiently.
Another aspect for the future is evaluating the algorithms on a larger scale with a more powerful machine and more time. Not only the $n, r, d$ and $k$ can be increased, but also the number of repetitions per graph. As a result, this would give a more precise picture about the properties of the algorithms, especially the value of the constant $C$ of \cref{eq:c_estimate}. Furthermore, even more combinations of ranks $r$ and degrees $d$ can be evaluated. Also, for more insight, the algorithms could be evaluated on, non-uniform graphs. 
From a theoretical perspective, the algorithms can be analyzed for improvement potential and bounds for the value of the constant can be investigated. Finally, the construction of random hypergraphs can be analyzed further and the used ideas combined in other ways.

